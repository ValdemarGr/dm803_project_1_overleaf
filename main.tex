\documentclass[oneside]{scrbook}
\setkomafont{author}{\scshape}
\usepackage{blindtext}
\usepackage[utf8]{inputenc}

\title{Avancerede Datastrukturer}
\subtitle{Projekt 1 DM803}
\author{Gabriella Juhl Jensen\thanks{CPR nr.: 160396, e-mail: gajen16} \and Samuel Valdemar Grange\thanks{CPR nr.: 081097, e-mail: sagra16} }
\subject{Skip lists and Scapegoat Trees}

\begin{document}

\maketitle
\chapter{Skip lists}
Generel about skip list...
\section{Investigateted}
\subsection{Avarage shearch complexity}
\subsection{Variation in shearch complexity}
\section{Test}
\chapter{Scapegoat Trees}
Genrel about scapegoat list...
Scapegoat Trees is a binary search tree that operate like a binarye search tree except of the balance of the tree is uphold by what we call $\alpha$-weight-balance. We have a $0.5\leq \alpha<1$ that we choose, to decide when the tree is $\alpha$-weight-balance. When we insert or delete a node such this is not the case anymore. 

\subsection{Introduction}
\subsection{Avarage shearch complexity}
\subsection{Variation in shearch complexity}
\subsection{frequensy and size}
\section{Test}
\chapter{Compare}
\chapter{Appendix}
\section{Code for skip Lists}
\section{Code for Scapegoat trees}


\end{document}
